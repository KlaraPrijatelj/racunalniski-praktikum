\documentclass[a4paper,10pt]{article}

% NALOGA 1
% Vključite pakete za podporo slovenščini,
% prepoznavo vhodnega kodiranja `utf8` in izhodnega kodiranja `T1`.
% Prekopirajte ustrezne tri vrstice iz pregledne datoteke `1-osnove.tex`.
\usepackage[utf8]{inputenc}   % vhodna datoteka je kodirana v sistemu UTF-8
\usepackage[T1]{fontenc}      % izhodna datoteka je korirana v sistemu T1
\usepackage[slovene]{babel}
\usepackage{amsmath}

\begin{document}

% Pri tej nalogi (in tudi vseh naslednjih) si pomagajte s paleto ukazov!
% Naslov, avtor, datum in povzetek naj bodo izdelani z uporabo ukazov in
% okolja, ki so v ta namen definirani v razredu `article`. Da se bo glava
% dokumenta s temi podatki prikazala, morate uporabiti ukaz `\maketitle`.

\title{Izdelava urnikov}
\author{Nejc Širovnik}
\date{14. 10. 2019}
\maketitle

% začetek povzetka
V članku so predstavljeni nekateri postopki za avtomatsko izdelavo urnikov. Opisani so
potrebni vhodni podatki, naštete so omejitve, ki jih je potrebno upoštevati pri izdelavi
urnikov, predstavljena je tudi metoda merjenja kakovosti izdelanega urnika.
% konec povzetka

\section{Program za izdelavo urnika}

Spominjam se osnovnošolskih in srednješolskih dni, ko smo vsi z nestrpnostjo pričakovali
začetek novega šolskega leta, ki pa je nemalokrat prinesel zelo hladen tuš ob pogledu na
natrpan urnik. Zakaj moramo v ponedeljek tako zgodaj vstati, ko pa se teden še ni niti
prav začel? Zakaj nam vsaj v petek niso malo popustili? Kdo je sploh lahko izdelal
takšen urnik?

To so vprašanja, na katera seveda nikoli nismo dobili odgovora, čeprav je le-ta na zadnje
vprašanje več kot očiten -- izdelala ga je oseba, ki so ji to delo naprtili. Naprtili je
zelo primerna beseda, saj gre za zahtevno opravilo, ki izdelovalcu urnikov nemalokrat
pusti ogromno sivih las. Ne bi se rad postavil v kožo ubožca, ki konča z izdelavo urnika,
nakar ves zgrožen ugotovi, da je spregledal prekrivanje dveh šolskih ur, kar pomeni, da
bi učenci morali biti pri dvoje urah hkrati. Tako mu seveda ne preostane drugega, kot da
poskusi znova ter po nekaj neuspelih poskusih vendarle izdela urnik, na katerega pa vseeno
leti kopica pripomb.

Naprednejše ročne metode izdelovanja urnikov potekajo s pomočjo modelov in tabel, vendar
tudi te pogosto ne vodijo do kakovostnih urnikov. Pri vseh je namreč prisoten faktor
omejenosti človeških možganov, ki se po hitrosti obdelave podatkov niti približno ne
morejo kosati z današnjimi procesorji.

Prav zmogljivost procesorjev je tista, ki že od izuma računalnika žene programerje k
ustvarjanju optimizacijskih algoritmov za iskanje kakovostnejšega urnika. Prav kakovost
pa je lahko dvorezen meč, saj nekateri urniki zadovoljijo potrebe učencev, a se premalo
osredotočajo na potrebe učiteljev, in obratno. Kateri urnik je potemtakem najkakovostnejši?
Ali tak urnik sploh obstaja?

Znano je, da so za izdelavo najzahtevnejši fakultetni urniki, zato od sedaj naprej
obravnavamo le slednje. Seznam omejitev, ki jih moramo upoštevati pri teh urnikih,
je namreč obsežnejši kot pri urnikih osnovnih in srednjih šol.

\subsection{Vhodni podatki}

Če želimo izdelati urnik, v prvi vrsti potrebujemo podatke, ki jih moramo upoštevati in
kasneje tudi obdelati. Te podatke preberemo iz podatkovnih baz ali drugih vrst datotek,
med katerimi vrh prestola zasedajo XML datoteke. Naslednji primer ponazarja, kako je v
XML datoteki predstavljen profesor Janez Novak, ki trikrat na teden predava predmet
matematika.

% Spodaj je primer vsebine XML datoteke, ki ga označite z okoljem `verbatim`.
% To okolje ohranja presledke na začetku vrstic, prelome vrstic, ter prikaže odstavek, 
% kot da je napisan na pisalni stroj.
% V tej datoteki so še trije načrti postopkov, ki jih tudi označite z okoljem `verbatim`.
% Namig: pomagajte si z iskanjem in zamenjavo komentarjev, kot so v spodnjem primeru.

% začetek okolja verbatim

\begin{verbatim}
   <profesor>
      <ime>Janez</ime>
      <priimek>Novak</priimek>
      <predmet>
         <imePredmeta>matematika</imePredmeta>
         <steviloUrNaTeden>3</steviloUrNaTeden>
      </predmet>
   </profesor>
\end{verbatim}
% konec okolja verbatim


% To popravite z ukazom `\noindent` ali pa tako, da odstranite prazno vrstico.
% (Nekoliko lepše izgleda, če v prazno vrstico vstavite % za komentar.)
Hierarhija značk omogoča enostavno vnašanje podatkov, hkrati pa poskrbi za preglednost.
Še enostavnejše je dostopanje do posameznih podatkov, saj večina programskih jezikov že
vsebuje metode za delo z XML datotekami.

Pri objektno orientiranih programskih jezikih (Delphi, C++, Java) vse vhodne podatke
opišemo s posameznimi razredi. Poglejmo, katere razrede je smiselno implementirati ter
katere podatke naj vsebujejo.

% Spodnji zamaknjeni odstavki so elementi neoštevilčenega seznama.
% S pomočjo palete ukazov naredite ustrezno okolje. 
% Z večimi kurzorji hkrati napišite še ukaze `\item`, kjer jih potrebujete.
% Pri prvem seznamu se vsi elementi začnejo z besedo `Razred`, zato lahko več 
% kurzorjev na pravih mestih dobite enostavno s pomočjo iskanja.

% V besedilu je definiranih nekaj novih pojmov -- zapisani so med poševnici. 
% Da bodo prikazani ležeče, uporabite ukaz `\emph`. 
% Kot pri razdelkih si pomagajte z orodjem _Replace_ in regularnimi izrazi -- 
% prav vam bo prišel isti vzorec za poljuben niz znakov, le da bo zdaj obdan z dvema poševnicama.

\begin{itemize}
	\item Razred /Profesor/ vsebuje osebne podatke profesorja (ime in priimek) ter
   podatke o predmetih, ki jih predava. Hkrati moramo za vsako uro v tednu vedeti, če
   je profesor takrat že zaseden, da mu slučajno ne bi dodelili dveh predavanj v istem
   časovnem terminu. To je najlažje preverjati s pomočjo tabele zasedenosti, ki je
   podrobneje opisana na koncu poglavja.

   \item Razred /Skupina/ predstavlja skupino študentov. Kot podatke torej potrebujemo
   smer skupine (npr. pedagoška matematika) ter število študentov v njej. Podobno kot pri
   profesorjih, tudi tukaj potrebujemo tabelo zasedenosti.

   \item Razred /Predavalnica/ nosi podatke o kapaciteti predavalnice, njeno oznako
   ter podatke o zasedenosti, ki so predstavljeni s tabelo.

   \item Razred /Predmet/ vsebuje podatke o posameznih predmetih. Kot podatke tega
   objekta je smiselno navesti ime predmeta ter tedensko število ur njegove izvedbe.
   Zanima nas tudi, katerim skupinam je predmet namenjen, ter v katerem semestru se bo
   predaval.
\end{itemize}


% To popravite z ukazom `\noindent` ali pa tako, da odstranite prazno vrstico.
% (Nekoliko lepše izgleda, če v prazno vrstico vstavite % za komentar.)
Kar pri treh razredih je kot podatek navedena tabela zasedenosti, zato si jo podrobneje
oglejmo. Vsako tabelo določata njena velikost ter tip vrednosti. Naslednji izračun nam
poda velikost tabele:
\[
\text{velikostTabele}=\text{številoUrNaDan}\cdot\text{številoDelovnihUr}\]
Če bi predpostavili, da študijski dan traja od sedme ure zjutraj do osmih zvečer in da
imamo pet delovnih dni, je velikost naše tabele 65 polj. Velikost te tabele torej pove,
koliko ur na teden obravnavamo. Ker se šolski tedni periodično ponavljajo, je seveda
dovolj, če podamo zasedenost le za en teden.


% Besede `bool`, `true`, `false` v spodnjem odstavku naj bodo napisane v pisavi 
% za izvorno kodo.
Tip vrednosti tabele ni striktno določen. Tako bi nekateri programerji uporabili
celoštevilsko tabelo, kjer bi vrednost 1 pomenila, da je oseba zasedena, vrednost 0
pa bi predstavljala njeno dosegljivost. Podobno bi postopali, če bi izbrali logični
tip (npr. bool v C++), kjer bi vrednost true predstavljala
dosegljivost, false pa zasedenost.

\subsection{Omejitve}

Programa za izdelavo (nekakovostnega) urnika načeloma ni težko narediti. Potrebujemo
le metodo, ki naključno prireja termine posameznim predmetom. Težava pri takšnem postopanju
je, da pri naključnem vstavljanju nismo upoštevali nobenih omejitev, prav zadoščanje omejitvam
pa je tisto, kar odlikuje dober urnik.

\subsubsection{Stroge omejitve}

Verjetno še niste videli profesorja, ki vsakih pet minut teče iz ene predavalnice v drugo,
ker so mu izdelalovalci urnika dodelili dve predavanji v istem časovnem terminu. Razlog
je preprost -- vsak urnik mora biti izdelan tako, da zadošča vsem strogim omejitvam.
Poglejmo si nekaj omejitev, ki jih urnik ne sme kršiti:

% Vsaka od vrstic v okolju `itemize` je en element naštevanja.
% začetek okolja itemize
   V urnik moramo razdeliti vse ure vseh predmetov.
   Noben profesor ne sme imeti več predavanj hkrati.
   Nobena skupina ne sme imeti več predavanj hkrati.
   Nobena predavalnica ne sme biti dodeljena večim predavanjem hkrati.
   Predavalnica mora imeti zadostno kapaciteto.
   Predavanja posebne vrste (npr. vaje iz računalništva) se morajo izvajati v ustrezni predavalnici.
   Urnik mora zadoščati zastavljenemu časovnemu okvirju (npr. od sedmih zjutraj do osmih zvečer).
% konec okolja itemize


% To popravite z ukazom `\noindent` ali pa tako, da odstranite prazno vrstico.
% (Nekoliko lepše izgleda, če v prazno vrstico vstavite % za komentar.)
Urnik, ki zadošča vsem strogim omejitvam, imenujemo zadosten urnik. Zadostni urnik tako ne
krši nobenega izmed zgoraj naštetih načel, a še zdaleč ne ustreza merilom kakovostnega urnika.

Algoritem je najbolje implementirati tako, da do kršenja strogih omejitev sploh ne more priti.
Tabela zasedenosti je dober primer pametne implementacije, saj tako že v osnovi  preprečujemo
dodeljevanje t.i. prekrivanj. Če želi algoritem dodeliti predavanje določeni predavalnici,
mora najprej v tabeli zasedenosti preveriti, če je v tistem terminu le-ta še na voljo,
sicer se vstavljanje ne izvede.

\subsubsection{Šibke omejitve}

V splošnem ne zahtevamo, da urnik zadošča vsem šibkim omejitvam, ker je to praktično neizvedljivo.
Resda pa so prav šibke omejitve tiste, ki določajo kakovost urnika, in dober algoritem bo
poskrbel, da se bo zadostilo večini le-teh. Naštejmo nekaj izmed njih:

% Vsaka od vrstic v okolju `itemize` je en element naštevanja.
% začetek okolja itemize
   Urnik naj vsebuje čim manj prostih ur za profesorje in študente.
   Študentje naj se čim manj selijo iz predavalnic.
   Predmet z veliko urami naj bo enakomerno porazdeljen čez celotno periodo (teden).
   Predavanja naj se izvajajo ob urah, ki so najugodnejše za poučevanje.
   V petek naj se študentom dodeli čim manj predavanj.
   Študentom se naj v času kosila dodeli prosta ura.
% konec okolja itemize


% To popravite z ukazom `\noindent` ali pa tako, da odstranite prazno vrstico.
% (Nekoliko lepše izgleda, če v prazno vrstico vstavite % za komentar.)
Obstaja še mnogo drugih šibkih omejitev, ki se jim je smiselno posvetiti. Nikomur najbrž ne
bi bilo preveč všeč, če bi uram športne vzgoje sledila še vrsta predavanj, saj bi tako
prepoteni in izčrpani študentje težko sledili tempu pouka.

Pri ročnem sestavljanju urnika se ponavadi sestavi zadostni urnik, ki zadošča le
najpomembnejšim šibkim omejitvam. Dobro zasnovan algoritem pa lahko preverja veliko
število šibkih omejitev, zato snovalci programov prav temu posvečajo ogromno pozornosti,
saj bo njihov končni produkt le tako zadovoljil profesorje in študente.

\subsection{Merjenje kakovosti}

Kakovost urnika se meri z zadoščanjem šibkim omejitvam, pri čemer že predpostavljamo
zadostnost urnika (zadoščanje strogim omejitvam). Ta način je smiseln predvsem zato,
ker si urnik, ki ne zadošča strogim omejitvam, tega naziva sploh ne zasluži.

Človek, ki najprej ustreli in šele nato razmisli, bi verjetno dejal, da število
zadoščenih šibkih omejitev določa kakovost urnika, kar pa je daleč od resnice. Medtem
ko so stroge omejitve po pomembnosti (skoraj) enakovredne, se šibke omejitve med seboj
krepko razlikujejo. Tako je bolj pomembno, da urnik vsebuje manj prostih ur za
študente in profesorje, kot pa je dodeljena prosta ura v času kosila. Kako bi torej
ovrednotili posamezne šibke omejitve?

V splošnem uporabljamo standardni postopek uteževanja omejitev, pri čemer vsaki šibki
omejitvi dodelimo celoštevilsko utež glede na njeno pomembnost. Vprašanje je tudi, kako
določiti to pomembnost. Če programer določa uteži po lastni presoji, so zagotovo zelo
subjektivno dodeljene. Dosti bolj smiselno je anketiranje, kjer bi študentje in profesorji
različnih izobraževalni ustanov ovrednotili posamezne šibke omejitve z ocenami od 1 do 10.
S tem bi nastavljene uteži pridobile na objektivnosti, verjetnost, da bo urnik dobro
sprejet s strani večine, pa bi se povečala.

Kot zanimivost naj omenimo, da se nekateri programerji odločajo tudi za utežitev strogih
omejitev s to razliko, da so vrednosti teh uteži izjemno velike, zato se bo algoritem
(skoraj) vedno "odločil", da jih ne bo prekršil. Tak način uteževanja pride v poštev
takrat, ko algoritem v osnovi ne zadosti strogim omejitvam urnika.

Ko enkrat ustrezno nastavimo vrednosti uteži, je treba ugotoviti, kako bomo postopali
v nadaljevanju. Algoritem za izdelavo urnikov je najbolje zasnovati tako, da po vsaki
spremembi izmeri njegovo kakovost. Tako moramo na vsakem koraku ugotavljati, katere
omejitve so bile kršene ter kolikokrat se kršitev pojavi. S ?? označimo število
kršitev omejitve ??, z ?? pa njeno utež. Z naslednjo formulo izračunamo
kakovost urnika:
??
Nižja vrednost formule pomeni kakovostnejši urnik, izračunamo pa jo na vsakem koraku.
Korak pri izdelavi urnika predstavlja eno izmed operacij nad njim (označuje njegovo
spremembo).

V 2. poglavju bomo opisali postopek gradnje urnika, saj bomo podrobneje spoznali
operacije nad urnikom in več vrst optimizacijskih metod, ki jih lahko brez oklevanja
poimenujemo kar osrčje programa za avtomatizirano izdelavo urnika.

\section{Avtomatizirana izdelava urnika}

V 1. poglavju smo izvedeli, da ročna izdela urnikov ni tako preprosta, kot bi si
na prvi pogled predstavljali. Še večji izziv predstavlja implementacija algoritma, ki da
(fakultetni) urnik. Obstaja pa velika razlika med prvim in drugim pristopom, saj ko je
algoritem enkrat dobro napisan, smo se težav z izdelovanjem urnika za vselej rešili.
Tako nas vsako leto čaka le spreminjanje vhodnih podatkov, ki jih algoritem sprejme in
nato iz njih izdela celovit urnik. Te vhodne podatke smo si skupaj s strogimi in šibkimi
omejitvami že podrobneje ogledali v prvem delu, tako da se lahko tokrat osredotočimo na
osrčje programa.

Za lažjo ilustracijo bi lahko potegnili kar nekaj vzporednic med programiranjem in -- ne
boste verjeli -- kuhanjem. Za kuhanje potrebujemo recept, ki je sestavljen iz dveh delov.
Prvi del predstavljajo sestavine, ki jih potrebujemo za kuhanje, glavni del pa predstavlja
postopek, kako iz teh sestavin pripraviti okusno kosilo. Postopki kuhe se med seboj
razlikujejo in nekateri pripeljejo do nekoliko slastnejšega obroka kot drugi.

Če se ob upoštevanju zgornjih opomb vrnemo k programiranju, sestavine v našem programu že
imamo, to so namreč podatki o profesorjih, predavalnicah, skupinah ter predmetih. Prav tako
imamo izbrane uteži za posamezne omejitve. Manjka le še glavni del recepta, ki podatke
ustrezno obdela ter iz njih sestavi urnik. Obstaja več postopkov oziroma metod, ki vodijo
do končnega urnika, podobno kot pri kuhi pa je tudi tukaj kakovost končnega izdelka odvisna
od izbire najustreznejše metode.

Končni produkt, ki ga bomo ustvarili s pomočjo kasneje omenjenih metod, bo urnik, ki je
predstavljen z množico /srečanj/. Posamezno srečanje je sestavljeno iz predmeta,
profesorja, ki predmet predava, ter skupin, ki jim je predmet namenjen. Srečanja so nato
vstavljena v termine predavalnic, tako da se za vsako srečanje ve, ob kateri uri in v
kateri predavalnici se bo izvajalo.

\subsection{Požrešna metoda}

Požrešna metoda je ena izmed strategij, s katerimi lahko rešujemo optimizacijske probleme -
to pomeni, da za dani problem iščemo najboljšo (optimalno) rešitev. Metoda na vsakem koraku
izmed vseh možnih rešitev izbere  tisto, ki daje (trenutno) največji dobiček oziroma najbolj
poveča vrednost kriterijske funkcije.

Podobno kot požrešna metoda nemalokrat postopamo tudi sami. Božični nakupi v trgovine
prinašajo dolge čakalne vrste in ponavadi se odločimo, da bomo z vozičkom stopili do blagajne,
na kateri čaka najmanj ljudi. Šele po polurnem polžjem približevanju blagajni ugotovimo,
da bi pri večini drugih blagajn prišli na vrsto prej, saj so tam ljudje opravljali le
manjše nakupe.

Znano je, da požrešna metoda pri nekaterih problemih odpove, je pa vseeno zelo uporabna
tehnika za iskanje rešitev, ko ne zahtevamo njihove optimalnosti. Urnik, ki ga dobimo s
pomočjo požrešne metode, je lahko namreč še vedno za nekaj nivojev boljši od ročno izdelanega.
Naslednje vrstice bodo pojasnile, kako implementirati postopek požrešne metode za primer urnika.

% začetek okolja verbatim

\begin{verbatim}
   (1) Sprehod po vseh predmetih.
   {
      (2) Sprehod po vseh predavalnicah.
      {
         (3) Računanje cene za proste termine.
         (4) Primerjava cene z najboljšo.
      }
      (5) Vstavljanje srečanja v urnik.
   }
\end{verbatim}
% konec okolja verbatim


% To popravite z ukazom `\noindent` ali pa tako, da odstranite prazno vrstico.
% (Nekoliko lepše izgleda, če v prazno vrstico vstavite % za komentar.)
Zgoraj predstavljena shema je le grob opis postopka, ki ga nad urnikom izvaja požrešna metoda,
zato si je smiselno podrobneje ogledati posamezne korake.

   Z zanko se sprehodimo po vektorju vseh predmetov. Predmet na ??-tem koraku
   označimo s ??.

   Naslednja vgnezdena zanka nas popelje po vseh predavalnicah. Predavalnico na
   ??-tem koraku označimo s ??.

   Za vse proste termine ?? izračunamo ceno vstavljanja
   ??, ki ga sestavlja ??, profesor, ki ta predmet predava,
   ter skupina, kateri je predmet namenjen.

   Če je trenutna cena boljša od najboljše dosedanje, potem najboljša dosedanja
   dobi vrednost trenutne. Hkrati si zapomnimo, za kateri termin smo izračunali ceno
   vstavljanja, da bomo ?? na koncu vstavili v termin z najboljšo
   ceno vstavljanja.

   Ko preverimo vse proste termine vseh predavalnic, vstavimo ??
   v termin z najboljšo ceno.

   Sedaj se ponovno vrnemo na začetek in pogledamo najugodnejši termin za naslednji
   predmet. Postopek ponavljamo, dokler se zunanja zanka, ki se sprehaja po vseh predmetih,
   ne zaključi. S tem smo namreč v urnik vstavili vse predmete.

\subsection{Optimizacijske metode}

Optimizacijske metode se že desetletja uporabljajo za reševanje optimizacijskih problemov.
Nekatere manj, druge bolj uspešno rešujejo zapletene probleme, pri čemer je eksponentno
naraščajoča zmogljivost procesorjev le voda na njihov mlin, saj tako lahko v krajšem
času izvedejo več operacij. Končne rešitve, ki jih dobimo z optimizacijskimi metodami,
so lokalni ali globalni optimumi.

Če želimo spoznati optimizacijske metode, se moramo najprej seznaniti z dvema pojmoma.
/Operacija/ je vsaka sprememba oziroma poseg v urnik med izvajanjem optimizacije.
Na primeru urnika si oglejmo, katere so smiselne operacije:

% Vsaka od vrstic v okolju `itemize` je en element naštevanja.
% začetek okolja itemize
   /Vstavljanje/ srečanja v urnik.
   /Brisanje/ srečanja iz urnika.
   /Zamenjava/ dveh srečanj v urnik.
% konec okolja itemize


% To popravite z ukazom `\noindent` ali pa tako, da odstranite prazno vrstico.
% (Nekoliko lepše izgleda, če v prazno vrstico vstavite % za komentar.)
/Okolica/ operacije je množica vseh možnih sprememb, ki jih lahko s to operacijo
opravimo na nekem koraku optimizacijskega postopka. Okolica zamenjave dveh srečanj v urniku
je torej množica vseh možnih zamenjav srečanj v trenutnem urniku.

/Sistem okolic/ je množica vseh možnih sprememb, ki jih lahko z vsemi operacijami iz
okolic sistema opravimo na nekem koraku optimizacijskega postopka. Oboroženi z novimi pojmi
se podajmo v naslednje vrstice.

\subsubsection{Lokalno vzpenjanje}

Lokalno vzpenjanje temelji na opazkah, da izbira širšega sistema okolic prinaša kakovostnejše
rešitve. Sistem okolic v primeru urnika predstavljajo prej omenjene operacije vstavljanja,
brisanja ter zamenjave srečanj. Osnovno lokalno vzpenjanje se teoretično sploh ne razlikuje
od požrešne metode, saj tudi pri tem postopku iz okolice vzamemo najboljšo rešitev. Kaj pa,
če bi lokalno vzpenjanje nadgradili?

Do sedaj smo vedno dopuščali le izbiro najboljše možne rešitve iz določene okolice, kar nas
je hitro pripeljalo do lokalnega maksimuma, iz katerega pa se je bilo nemogoče izviti, saj
je najboljša rešitev iz okolice lokalnega maksimuma ponavadi kar lokalni maksimum sam.
Nadgradnja lokalnega vzpenjanja je jasno razvidna iz naslednje sheme.


\begin{verbatim}
   % začetek okolja verbatim
   (1) Izvajaj dokler...
   {
      (2) Iz sistema okolic izberi okolico.
      (3) Iz okolice izberi spremembo.
      (4) Izvedi spremembo.
   }
   % konec okolja verbatim
\end{verbatim}


% To popravite z ukazom `\noindent` ali pa tako, da odstranite prazno vrstico.
% (Nekoliko lepše izgleda, če v prazno vrstico vstavite % za komentar.)
Nadobudnežem priporočamo, da najprej sami razmislijo, kaj je globja vsebina posameznih
korakov, in si šele nato ogledajo njihov natančen opis.

% začetek okolja enumerate
   Zunanjo zanko izvajamo, dokler ne dosežemo enega izmed ustavitvenih pogojev.
   Če nekaj časa kljub spreminjanju operacij nad urnikom vedno znova obstojimo na isti
   rešitvi, potem to najverjetneje pomeni, da smo prišli do lokalnega ekstrema. Drugi
   ustavitveni pogoj je zadovoljivo velika vrednost kriterijske funkcije.

   Naključno izberemo okolico trenutne rešitve. Z okolico izberemo tudi operacijo,
   ki jo izvajamo na tem koraku (npr. zamenjava dveh srečanj).

   Če v tem koraku izberemo tisto zamenjavo srečanj, ki izmed vseh možnih najbolj
   poveča vrednost kriterijske funkcije, potem postopamo enako, kot požrešna metoda. Če
   želimo nadgraditi postopek lokalnega vzpenjanja, ne potrebujemo nujno najboljše možne
   zamenjave, ampak prvo, ki poveča vrednost kriterijske funkcije. Izjemoma dovolimo tudi
   tisto zamenjavo, ki vrednost kriterijske funkcije nekoliko zmanjša, kar se na prvi
   pogled zdi nesmiselno, nas pa dolgoročno pripelje do boljših rešitev.

   Izbrano zamenjavo dveh srečanj izvedemo na trenutnem urniku. Vidimo, da en korak
   lokalnega vzpenjanja res predstavlja eno spremembo nad urnikom, v konkretnem primeru
   smo izvedli (eno) zamenjavo dveh srečanj.
% konec okolja enumerate


% To popravite z ukazom `\noindent` ali pa tako, da odstranite prazno vrstico.
% (Nekoliko lepše izgleda, če v prazno vrstico vstavite % za komentar.)
Na tak način izboljšana metoda lokalnega vzpenjanja je časovno veliko zahtevnejša od
požrešne metode, saj se vrednost kriterijske funkcije ne povečuje tako drastično. Vseeno
pa vidimo, da se počasi približujemo lokalnemu maksimumu in sčasoma nas bo postopek
pripeljal do njega. Naključna izbira okolic ter sprejemanje ne nujno najboljše rešitve
nas varujeta, da ne bomo obstali v prvem lokalnem maksimumu, ampak bomo vedno znova
iskali boljše rešitve.

Slabost do sedaj obravnavanih tehnik še vedno ostaja -- ko enkrat obtičimo v lokalnem
maksimumu, se iz njega ne moremo rešiti. Naslednja metoda poskuša na svojevrsten, a zelo
sistematičen način odpraviti to težavo.

\subsubsection{Iskanje s tabuji}

Že samo ime spominja na prepovedi, ki jih le-ta skriva v svojem naročju. Kaj sploh prepovedujemo?
Glede na to, da bi naj iskanje s tabuji odpravilo problem vračanja v isti lokalni maksimum,
prepovedujemo izvajanje sprememb, s katerimi bi se vrnili v že najden lokalni maksimum.

Rešitev se ponuja kar sama -- za naslednjih ?? korakov prepovemo vsako spremembo, ki jo
opravimo v koraku optimizacije. Izbira konstante ?? je odvisna od samega problema, saj
ne obstaja neki splošen ??, ki bi nas v vsakem primeru pripeljal iz lokalnega ekstrema.
Imamo torej seznam prepovedanih sprememb -- /tabujev/ in na vsakem koraku za izbrano
spremembo preverimo, če je element seznama tabujev. Če spremembe ni v seznamu tabujev, jo
preprosto izvedemo, v primeru, da je, pa postopek vrnemo na iskanje druge spremembe.
Algoritem iskanja s tabuji se od lokalnega vzpenjanja torej razlikuje le v tem, da pred
izbiro spremembe preveri, če je vsebovana v seznamu prepovedanih sprememb. Poglejmo, zakaj
je takšno postopanje smiselno.

Naj bo ?? ter ?? sprememba urnika, ki nas je pripeljala v lokalni maksimum. To
spremembo (npr. zamenjava ?? in ??) sedaj shranimo v
seznam tabujev. V naslednjem koraku izvedemo spremembo ??, ki se prav tako shrani v
seznam prepovedanih sprememb. Podobno naredimo za vse spremembe v nadaljnih korakih. S
tem poskrbimo, da se lahko v isti lokalni maksimum vrnemo šele čez ?? korakov, saj
se sprememba ??, ki vodi do lokalnega maksimuma, šele tedaj izbriše iz seznama
prepovedanih sprememb.

Če smo dosegli lokalni maksimum in vrednost kriterijske funkcije v ?? korakih uspemo
rešiti iz okolice tega lokalnega maksimuma, potem se je metoda iskanja s tabuji izkazala
kot uspešna.

\subsection{Končni algoritem}

Za konec si oglejmo, kako je videti optimizacijski del algoritma za iskanje najboljšega
urnika, pri čemer je najboljši urnik tisti, ki s kršenjem omejitev pridobi najmanj negativnih
točk, oziroma ki ima največjo vrednost kriterijske funkcije.

% začetek okolja verbatim
(1) r = rešitev, dobljena s požrešno metodo.
(2) Izvajaj, dokler...
{
   (3) Z optimizacijo izboljšaj r.
   (4) Oceni r.
}
% začetek okolja verbatims

   V prvem koraku izvedemo požrešno metodo, ki rešitev urnika pripelje v lokalni maksimum.

   Zanko izvajamo, dokler ne dosežemo enega izmed ustavitvenih pogojev.

   Rešitev ??, ki smo jo dobili s pomočjo požrešne metode, izboljšamo z lokalnim vzpenjanjem
   ali iskanjem s tabuji. Kot je že bilo povedano, je iskanje s tabuji nekoliko primernejša metoda,
   saj nas bo lažje pripeljala iz lokalnega maksimuma, do katerega smo prišli s požrešno metodo.
   Vsekakor pa ni nič narobe, če srečo poskusimo tudi z lokalnim vzpenjanjem.

   Na vsakem koraku ocenimo izboljšamo rešitev ??. Če je vrednost kriterijske funkcije
   za trenutno rešitev prešla zastavljeno zadovoljivo mejo, potem se program zaključi. Prav tako
   se zaključi, če po ?? korakih še vedno tičimo v istem lokalnem maksimumu.


% To popravite z ukazom `\noindent` ali pa tako, da odstranite prazno vrstico.
% (Nekoliko lepše izgleda, če v prazno vrstico vstavite % za komentar.)
To je le ena izmed možnosti, kako zastaviti optimizacijski potek, s katero smo želeli predstaviti
simbiozo opisanih metod. Nič ne bi bilo narobe, če bi za iskanje najboljšega urnika posamično
uporabili katero izmed tehnik -- mogoče bi katero izmed teh postopanj dalo celo boljši rezultat.

\subsection{Zaključek}

% V spodnjem odstavku naj bodo črke G, U in I v "Graphical User Interface" podčrtane.
Sledili smo receptu in s tem izdelali "okusen obrok", treba ga je le še ustrezno postreči
gostom za mizo. S tem imamo v mislih grafični uporabniški vmesnik (Graphical
User Interface), ki mora urnik uporabnikom predstaviti na razumljiv in
pregleden način. Končnega izgleda izdelka nikakor ne smemo zanemariti, saj je ta med uporabniki
vreden skoraj toliko kot kakovosten urnik. Odločili smo se, da za razmišljanje o tej temi vaši
domišljiji pustimo prosto pot.

Spoznali smo torej tri metode za reševanje optimizacijskih problemov, pri čemer pa niti za eno
ne moremo z gotovostjo trditi, da nas bo pripeljala do globalnega maksimuma. To nekako opozarja
na kompleksnost problemov, hkrati pa tudi na neraziskanost tega področja. Verjamemo, da bo
prihodnost prinesla postopke, ki bodo s pomočjo tedanjih več tisoč-jedrnih procesorjev znali
poiskati globalne rešitve za nekatere današnje optimizacijske probleme. Prihodnost pa bo
hkrati prinesla tudi zahtevnejše probleme in s tem nove izzive za velike ume tistega časa.
\end{document}
